\documentclass{article}

\usepackage[margin=.4in]{geometry}

\usepackage[T1]{fontenc}
\usepackage{lmodern}
\usepackage[scaled=0.82]{beramono}
\usepackage{microtype}

\usepackage{tikz}
\usetikzlibrary{arrows}
\usetikzlibrary{arrows.meta}

\usepackage{cmvec}

% For demoing code
\usepackage{showexpl}
\usepackage{xcolor}
% http://tex.stackexchange.com/questions/17646/fixed-width-font-with-ltxexample-environment
\sbox0{\small\ttfamily A}
\edef\mybasewidth{\the\wd0 }
\lstset{explpreset={
  rframe={},
  xleftmargin=.25in,
  columns=flexible,
  language={TeX},
  basicstyle=\small\ttfamily,
  columns=fixed,
  basewidth=\mybasewidth,
  backgroundcolor=\color{gray!30},
  keywordstyle=\relax,
}}

\usepackage{blindtext}

%\CMIndexedSymbol{MS}{X}
%\CMIndexedSymbol{ms}{x}

\pagestyle{empty}


\CMIndexedSymbol{MS}{X}
\CMIndexedSymbol{ms}{x}

\begin{document}


\setlength{\parindent}{0pt}

Make \verb+cmvec.sty+ available to your \LaTeX\
installation. A simple way to do this is to copy \verb+cmvec.sty+ into
the same directory as your source \LaTeX\ document. Then, add the following
to your preamble:
\begin{center}
\verb+\usepackage{cmvec}+
\end{center}

The \verb+cmvec+ package requires a recent \LaTeX\ distribution. \verb+TikZ+
must be version 3.0.0 or higher (which came out in 2013/12). The implementation
is a dirty hack. The goal is to provide uniform way to annotate symbols with
vector marks and also provide convenient subscripting. The solutions here are
\emph{not} nicely implemented and will not be robust to different fonts, etc.

Unfortunately, there are still many hurdles to an arbitrary, extensible
accenting library on characters. So this package currently has two separate
implementations. The first is a pure-\LaTeX\ hack that does not provide
extensible. The second is a \verb+TikZ+-based solution. Both are provided
through the same API. Here are the macros that you'll need:

\begin{enumerate}
  \item \verb+\CMIndexedSymbol[options]{macroname}{symbol}+
  \item \verb+\CMSuperIndexedSymbol[options]{macroname}{symbol}{superscript}+
\end{enumerate}

Each of these macros defines a macro \verb+\macroname+ that supports
left, right, and leftright vector symbols and also provides Python-like
indexing.  These commands can be used anywhere in the document, but typically,
one should declare them just once in the preamble. The \verb+options+ argument
should be a valid arrow type:
\verb+arrow+, \verb+harpoon+, \verb+xarrow+, and \verb+xharpoon+; the default
value is \verb+xharpoon+. The \verb+x+ variants represent extensible vector
symbols, and for these variants only, one can additionally pass in a \verb+style=+
parameter along with the arrow type to customize how the arrow is drawn. This
can be used to customize the shortening and arrow size. The macro that is
created by this command supports
the following syntax:

\begin{center}
\verb+\macroname(direction)[start][end]+
\end{center}

Each argument \verb+(direction)+, \verb+[start]+ \verb+[end]+ is
optional. Valid directions are \verb+<+, \verb+>+, and \verb+<>+ for left, right,
and leftright directions, respectively. This is best explained through example:

\begin{LTXexample}[pos=b]
% Define the indexed symbol
\CMIndexedSymbol[harpoon]{MS}{X}

% Now use it
$\begin{matrix*}[l]
  \MS & \MS[0] & \MS[0][L] & \MS[][L] & \MS[L][]\\
  \MS(<) & \MS(>) & \MS(<>) & \MS(<)[0] & \MS(<>)[t]
\end{matrix*}$
\end{LTXexample}

Below are demonstrations of each arrow type. Because of the differing
implementations, the extensible arrows do not look like the non-extensible
arrows.\\

\begin{tabular}{rcrc}
\verb+arrow+
&
\CMIndexedSymbol[arrow]{MS}{X}\CMIndexedSymbol[arrow]{MSXX}{XX}
\CMIndexedSymbol[arrow]{ms}{x}\CMIndexedSymbol[arrow]{msXX}{xx}
$\begin{matrix}
\MS     & \MS[3]     & \MS[3][5]     & \MS[][5]     & \MS[5][]     & \MSXX[0]\\
\MS(<)  & \MS(<)[3]  & \MS(<)[3][5]  & \MS(<)[][5]  & \MS(<)[5][]  & \MSXX(<)[0]\\
\MS(>)  & \MS(>)[3]  & \MS(>)[3][5]  & \MS(>)[][5]  & \MS(>)[5][]  & \MSXX(>)[0]\\
\MS(<>) & \MS(<>)[3] & \MS(<>)[3][5] & \MS(<>)[][5] & \MS(<>)[5][] & \MSXX(<>)[0]\\
\ms     & \ms[3]     & \ms[3][5]     & \ms[][5]     & \ms[5][]     & \msXX[0]\\
\ms(<)  & \ms(<)[3]  & \ms(<)[3][5]  & \ms(<)[][5]  & \ms(<)[5][]  & \msXX(<)[0]\\
\ms(>)  & \ms(>)[3]  & \ms(>)[3][5]  & \ms(>)[][5]  & \ms(>)[5][]  & \msXX(>)[0]\\
\ms(<>) & \ms(<>)[3] & \ms(<>)[3][5] & \ms(<>)[][5] & \ms(<>)[5][] & \msXX(<>)[0]\\
\end{matrix}$
&
\verb+harpoon+
&
\CMIndexedSymbol[harpoon]{MS}{X}\CMIndexedSymbol[harpoon]{MSXX}{XX}
\CMIndexedSymbol[harpoon]{ms}{x}\CMIndexedSymbol[harpoon]{msXX}{xx}
$\begin{matrix}
\MS     & \MS[3]     & \MS[3][5]     & \MS[][5]     & \MS[5][]     & \MSXX[0]\\
\MS(<)  & \MS(<)[3]  & \MS(<)[3][5]  & \MS(<)[][5]  & \MS(<)[5][]  & \MSXX(<)[0]\\
\MS(>)  & \MS(>)[3]  & \MS(>)[3][5]  & \MS(>)[][5]  & \MS(>)[5][]  & \MSXX(>)[0]\\
\MS(<>) & \MS(<>)[3] & \MS(<>)[3][5] & \MS(<>)[][5] & \MS(<>)[5][] & \MSXX(<>)[0]\\
\ms     & \ms[3]     & \ms[3][5]     & \ms[][5]     & \ms[5][]     & \msXX[0]\\
\ms(<)  & \ms(<)[3]  & \ms(<)[3][5]  & \ms(<)[][5]  & \ms(<)[5][]  & \msXX(<)[0]\\
\ms(>)  & \ms(>)[3]  & \ms(>)[3][5]  & \ms(>)[][5]  & \ms(>)[5][]  & \msXX(>)[0]\\
\ms(<>) & \ms(<>)[3] & \ms(<>)[3][5] & \ms(<>)[][5] & \ms(<>)[5][] & \msXX(<>)[0]\\
\end{matrix}$\\
&&&\\
\verb+xarrow+
&
\CMIndexedSymbol[xarrow]{MS}{X}\CMIndexedSymbol[xarrow]{MSXX}{XX}
\CMIndexedSymbol[xarrow]{ms}{x}\CMIndexedSymbol[xarrow]{msXX}{xx}
$\begin{matrix}
\MS     & \MS[3]     & \MS[3][5]     & \MS[][5]     & \MS[5][]     & \MSXX[0]\\
\MS(<)  & \MS(<)[3]  & \MS(<)[3][5]  & \MS(<)[][5]  & \MS(<)[5][]  & \MSXX(<)[0]\\
\MS(>)  & \MS(>)[3]  & \MS(>)[3][5]  & \MS(>)[][5]  & \MS(>)[5][]  & \MSXX(>)[0]\\
\MS(<>) & \MS(<>)[3] & \MS(<>)[3][5] & \MS(<>)[][5] & \MS(<>)[5][] & \MSXX(<>)[0]\\
\ms     & \ms[3]     & \ms[3][5]     & \ms[][5]     & \ms[5][]     & \msXX[0]\\
\ms(<)  & \ms(<)[3]  & \ms(<)[3][5]  & \ms(<)[][5]  & \ms(<)[5][]  & \msXX(<)[0]\\
\ms(>)  & \ms(>)[3]  & \ms(>)[3][5]  & \ms(>)[][5]  & \ms(>)[5][]  & \msXX(>)[0]\\
\ms(<>) & \ms(<>)[3] & \ms(<>)[3][5] & \ms(<>)[][5] & \ms(<>)[5][] & \msXX(<>)[0]\\
\end{matrix}$
&
\verb+xharpoon+
&
\CMIndexedSymbol[xharpoon]{MS}{X}\CMIndexedSymbol[xharpoon]{MSXX}{XX}
\CMIndexedSymbol[xharpoon]{ms}{x}\CMIndexedSymbol[xharpoon]{msXX}{xx}
$\begin{matrix}
  \MS     & \MS[3]     & \MS[3][5]     & \MS[][5]     & \MS[5][]     & \MSXX[0]\\
  \MS(<)  & \MS(<)[3]  & \MS(<)[3][5]  & \MS(<)[][5]  & \MS(<)[5][]  & \MSXX(<)[0]\\
  \MS(>)  & \MS(>)[3]  & \MS(>)[3][5]  & \MS(>)[][5]  & \MS(>)[5][]  & \MSXX(>)[0]\\
  \MS(<>) & \MS(<>)[3] & \MS(<>)[3][5] & \MS(<>)[][5] & \MS(<>)[5][] & \MSXX(<>)[0]\\
  \ms     & \ms[3]     & \ms[3][5]     & \ms[][5]     & \ms[5][]     & \msXX[0]\\
  \ms(<)  & \ms(<)[3]  & \ms(<)[3][5]  & \ms(<)[][5]  & \ms(<)[5][]  & \msXX(<)[0]\\
  \ms(>)  & \ms(>)[3]  & \ms(>)[3][5]  & \ms(>)[][5]  & \ms(>)[5][]  & \msXX(>)[0]\\
  \ms(<>) & \ms(<>)[3] & \ms(<>)[3][5] & \ms(<>)[][5] & \ms(<>)[5][] & \msXX(<>)[0]\\
\end{matrix}$
\end{tabular}\\

Note that typically one will not use both vector symbols and double indexing.
That is, it makes no sense to write $\MS(<)[0][3]$ since you might as well have
written $\MS[0][3]$.  On the other hand, it can make sense to use both
notations for realizations of random vectors---suppose you want to take elements $0$ to $3$ from
a specific semi-infinite future: $\ms(>)[0][3]$.\\

Here an inline demonstration. Note that interline spacing is legible.\\

$XyXyXyXyXyXyXyXyXyXyXyXyXyXyXyXyXyXyXyXyXyXyXyXyXyXyXyXyXy$\\
$XyXyXyXyXyXyXyXyXyXyXyXyXyXyXyXyXyXyXyXyXyXyXyXyXyXyXyXyXy$\\
$XyXyXyXyXyXyXyXyXyXyXyXyXyXyXy%
\MS(<) \, \MS(>) \, \MS(<>) \, \MS[0][3] \, \MS(<)[5]%
XyXyXyXyXyXyXyXyXy$
\\
$XyXyXyXyXyXyXyXyXyXyXyXyXyXyXyXyXyXyXyXyXyXyXyXyXyXyXyXyXy$\\
$XyXyXyXyXyXyXyXyXyXyXyXyXyXyXyXyXyXyXyXyXyXyXyXyXyXyXyXyXy$\\[.2in]

Suppose you plan on indexing both $X$ and $Y$, then you define both:
\begin{LTXexample}[pos=b]
\CMIndexedSymbol{MSi}{X} % input
\CMIndexedSymbol{MSo}{Y} % output
$\lim_{L \to \infty} I[ \MSi[0][L] : \MSo[0][L] ] \stackrel{?}{=} I[ \MSi(<)[t] : \MSo(>)[t]]$
\end{LTXexample}\vspace{.1in}

It may also be helpful to freeze arrow directions to certain macro names:
\begin{LTXexample}[pos=b]
\CMIndexedSymbol{MS}{X}
\CMIndexedSymbol{ms}{x}
\newcommand{\Past}{\MS(<)}
\newcommand{\past}{\ms(<)}
\newcommand{\Future}{\MS(>)}
\newcommand{\future}{\ms(>)}
\begin{align*}
  \Past[0]   &= \cdots \MS[-3] \MS[-2] \MS[-1] &
  \Future[0] &= \MS[0] \MS[1]  \MS[2]  \cdots \\
  \past[0]   &= \cdots \ms[-3] \ms[-2] \ms[-1] &
  \future[0] &= \ms[0] \ms[1]  \ms[2]  \cdots \\
\end{align*}
\end{LTXexample}

For the extensible varieties, you can customize the shortening and arrowhead
at macro definition time as follows:
\begin{LTXexample}[pos=b]
\CMIndexedSymbol[xharpoon, style=red]{msia}{i}
\CMIndexedSymbol[xharpoon, style=red]{msib}{\,i\,}
\CMIndexedSymbol[xarrow, style={-{>[length=1mm]}, shorten <=0pt, shorten >=0pt, green}]{msic}{i}
\CMIndexedSymbol[harpoon]{msid}{i}
$\msia(<>) \: \msib(<>) \: \msic(<>) \: \msid(<>)$
\end{LTXexample}


This is sometimes necessary since the length of the arrows is created with
\verb+TikZ+, which does not account for font skewness.



That's pretty much it.\\



\newpage

The package defines a number of other lower-level
commands that \emph{might} be of more general use, but probably not.
These are described now.

\begin{itemize}

\item Proper argmin and argmax:
\begin{LTXexample}[pos=b]
\begin{align*}
  &\arg\min_x (2x^2 - 3x + 5) \\
  &\argmin_x (2x^2 - 3x + 5)
\end{align*}
\end{LTXexample}

\item For summations with wide subscripts\ldots
\begin{LTXexample}[pos=b]
\begin{align*}
  A &= \sum_{i, j \in B_{ij}} X_i^j\\
  A &= \sum_{\mathclap{i, j \in B_{ij}}} X_i
\end{align*}
\end{LTXexample}

\item When you want to center math within a box whose width is
specified by other math.
\begin{LTXexample}[pos=b]
$aaabbbccc$\\
$aaa\phantomword[c]{bbb}{Q}ccc$
\end{LTXexample}

\item A customizable vector symbol. Macros should use this.
\begin{LTXexample}[pos=b]
\CMvector[symbol=\leftarrow]{X} \quad
\CMvector[symbol=\leftarrow, pre=\Large\textcolor{red}, ]{X} \quad
\CMvector[symbol=\leftarrow, post=*]{X} \quad
\CMvector[symbol=\leftarrow, raise=1.8]{X}
\end{LTXexample}

\item Although \verb+\leftharpoonup+ and \verb+\rightharpoonup+ exist, there
is no left-right harpoon. Here is a customized version that combines the
ones that do exist. The spacing is hardcoded and probably will only look good
with certain fonts.
\begin{LTXexample}[pos=b]
$\leftharpoonup \: \leftrightharpoonup \: \rightharpoonup$
\end{LTXexample}

\item Convenience functions that use \verb+\CMvector+.
\begin{LTXexample}[pos=b]
\CMlarrow{X} \: \CMlrarrow{X} \: \CMrarrow{X} \\
\CMlharpoon{X} \: \CMlrharpoon{X} \: \CMrharpoon{X} \\
\end{LTXexample}

\item A macro that defines specialized vector symbols that make use of Python index notation
and has clean support for the vector direction, depending on what notation you want to use.
\begin{LTXexample}[pos=b]
\CMIndexedSymbol{MS}{X}
\[\begin{matrix}
\MS     & \MS[3]     & \MS[3][5]     & \MS[][5]     & \MS[5][]     \\
\MS(<)  & \MS(<)[3]  & \MS(<)[3][5]  & \MS(<)[][5]  & \MS(<)[5][]  \\
\MS(>)  & \MS(>)[3]  & \MS(>)[3][5]  & \MS(>)[][5]  & \MS(>)[5][]  \\
\MS(<>) & \MS(<>)[3] & \MS(<>)[3][5] & \MS(<>)[][5] & \MS(<>)[5][] \\
\end{matrix}\]

\CMIndexedSymbol[arrow]{MS}{X}
\[\begin{matrix}
\MS     & \MS[3]     & \MS[3][5]     & \MS[][5]     & \MS[5][]     \\
\MS(<)  & \MS(<)[3]  & \MS(<)[3][5]  & \MS(<)[][5]  & \MS(<)[5][]  \\
\MS(>)  & \MS(>)[3]  & \MS(>)[3][5]  & \MS(>)[][5]  & \MS(>)[5][]  \\
\MS(<>) & \MS(<>)[3] & \MS(<>)[3][5] & \MS(<>)[][5] & \MS(<>)[5][] \\
\end{matrix}\]
\end{LTXexample}

\item Typically, you'll want to set up a few of these for regular use:
\begin{LTXexample}[pos=b]
% Put this in preamble somewhere
\CMIndexedSymbol[harpoon]{MS}{X}
\CMIndexedSymbol[harpoon]{ms}{x}
\CMSuperIndexedSymbol[arrow]{FCS}{S}{+}

% Some familiar commands
\newcommand{\BiInfinity}{\MS(<>)}
\newcommand{\biinfinity}{\ms(<>)}
\newcommand{\Past}{\MS(<)}
\newcommand{\past}{\ms(<)}
\newcommand{\Future}{\MS(>)}
\newcommand{\future}{\ms(>)}

% Now we can use them
\begin{displaymath}
\begin{matrix}
  \BiInfinity & \biinfinity & \Past & \past & \Future & \future \\
  \FCS[0] & \FCS(>)[3] & \MS[0][3] & \past[3] & \MS(<>)[3] & \Future[3]
\end{matrix}
\end{displaymath}
\end{LTXexample}

\item Notation for single symbol, range of symbols. Two different options
for semi-infinite sequences. If you might be using bi-infinite sequences,
it is recommended you use the second option for semi-infinite sequences.
\begin{LTXexample}[pos=b]
\CMIndexedSymbol[harpoon]{MS}{X}
\begin{align*}
  \MS         &&& \text{symbol}\\
  \MS[t]      &&& \text{symbol at time } t\\
  \MS[-1][3]  &&& \MS[-1] \MS[0] \MS[1] \MS[2]\\
  &&&\\
  \MS[][3]    &&& \cdots \MS[0] \MS[1] \MS[2] \\
  \MS[3][]    &&& \MS[3] \MS[4] \MS[5] \cdots\\
  &&&\\
  \MS(<)[3]   &&& \cdots \MS[0] \MS[1] \MS[2] \\
  \MS(>)[3]   &&& \MS[3] \MS[4] \MS[5] \cdots\\
  \MS(<>)[3]  &&& \MS(<)[3]\MS(>)[3]
\end{align*}
\end{LTXexample}

\end{itemize}

\end{document}
