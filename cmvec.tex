\documentclass{article}

\usepackage[margin=.5in]{geometry}

\usepackage[T1]{fontenc}
\usepackage{lmodern}
\usepackage[scaled=0.82]{beramono}
\usepackage{microtype}

\usepackage{tikz}
\usetikzlibrary{arrows}
\usetikzlibrary{arrows.meta}

\usepackage{cmvec}

\usepackage{accents}
\newcommand{\vect}[1]{\accentset{\leftharpoonup}{#1}}
\newcommand{\vvect}[1]{\accentset{\xleftrightharpoonup{#1}}{#1}}
\newcommand{\vvvect}[2][]{\accentset{\myarrow[#1]{#2}}{#2}}

% For demoing code
\usepackage{showexpl}
\usepackage{xcolor}
% http://tex.stackexchange.com/questions/17646/fixed-width-font-with-ltxexample-environment
\sbox0{\small\ttfamily A}
\edef\mybasewidth{\the\wd0 }
\lstset{explpreset={
  rframe={},
  xleftmargin=.25in,
  columns=flexible,
  language={TeX},
  basicstyle=\small\ttfamily,
  columns=fixed,
  basewidth=\mybasewidth,
  backgroundcolor=\color{gray!30},
  keywordstyle=\relax,
}}

\def\showmeaning#1{\ttfamily\small\meaning#1}

% Overline that accounts for italics.
% http://tex.stackexchange.com/a/95084/6896
\newcommand{\myol}[2][3]{{}\mkern#1mu\overline{\mkern-#1mu#2}}

\newcommand{\myarrow}[2][]{%
  \begin{tikzpicture}[baseline, inner sep=0pt, outer sep=0pt]%
    \node[anchor=base] (arg) {$\hphantom{#2}$};%
    \draw[-{Stealth[length=.7mm]}, shorten <= 1pt, shorten >= 0pt, #1]
         (arg.west) -- (arg.east);%
  \end{tikzpicture}%
}

\newcommand\xleftrightharpoonup[1]{%
\begin{tikzpicture}[baseline=(arg.base), inner sep=0pt, outer sep=0pt]
\node[anchor=base] (arg) {$#1$};
\draw[-{Stealth[harpoon,swap,length=1mm]},shorten >= 0.5pt] (arg.north east) -- (arg.north west);
\end{tikzpicture}%
}

\usepackage{blindtext}


\def\makemacro#1#2#3{%
  \def#1(##1){{\makemacroK{#3}^{\csname\ifx##1<left\else right\fi#2\endcsname}}}%
}
\def\makemacroA{\def\makemacroI{}\futurelet\next\makemacroB}
\def\makemacroB{\ifx\next[\expandafter\makemacroC\fi}
\def\makemacroC[#1]{\def\makemacroI{#1}\futurelet\next\makemacroD}
\def\makemacroD{\ifx\next[\expandafter\makemacroE\else_{\makemacroI}\fi}
\def\makemacroE[#1]{_{\makemacroI:#1}}
\def\makemacroK#1{\mathop{%
  \setbox0=\hbox{$#1_0$}\setbox2=\hbox{$#1\null_0$}%
  #1\kern\wd0\kern-\wd2}\limits
}

\makemacro \Qaz {arrow}{Q}
\makemacro \Qwe {harpoonup}{W}

\CMIndexedSymbol{MS}{X}
\CMIndexedSymbol{ms}{x}
\CMIndexedSymbol{msf}{f}
\CMIndexedSymbol{msd}{d}
\CMIndexedSymbol{msh}{h}
\CMIndexedSymbol{msi}{i}



\begin{document}
\setlength{\parindent}{0pt}
$
\CMrarrow{X} \: \CMrarrow{x} \:
\CMrarrow{Y} \: \CMrarrow{y} \:
\CMrarrow{Z} \: \CMrarrow{z} \:
\CMxrarrow{X} \: \CMxrarrow{x} \:
\CMxrarrow{Y} \: \CMxrarrow{y} \:
\CMxrarrow{Z} \: \CMxrarrow{z}
$ \\
$
\CMlarrow{X} \: \CMlarrow{x} \:
\CMlarrow{Y} \: \CMlarrow{y} \:
\CMlarrow{Z} \: \CMlarrow{z} \:
\CMxlarrow{X} \: \CMxlarrow{x} \:
\CMxlarrow{Y} \: \CMxlarrow{y} \:
\CMxlarrow{Z} \: \CMxlarrow{z}
$ \\
$
\CMlrarrow{X} \: \CMlrarrow{x} \:
\CMlrarrow{Y} \: \CMlrarrow{y} \:
\CMlrarrow{Z} \: \CMlrarrow{z} \:
\CMxlrarrow{X} \: \CMxlrarrow{x} \:
\CMxlrarrow{Y} \: \CMxlrarrow{y} \:
\CMxlrarrow{Z} \: \CMxlrarrow{z}
$ \\
$
\CMrharpoon{X} \: \CMrharpoon{x} \:
\CMrharpoon{Y} \: \CMrharpoon{y} \:
\CMrharpoon{Z} \: \CMrharpoon{z} \:
\CMxrharpoon{X} \: \CMxrharpoon{x} \:
\CMxrharpoon{Y} \: \CMxrharpoon{y} \:
\CMxrharpoon{Z} \: \CMxrharpoon{z}
$ \\
$
\CMlharpoon{X} \: \CMlharpoon{x} \:
\CMlharpoon{Y} \: \CMlharpoon{y} \:
\CMlharpoon{Z} \: \CMlharpoon{z} \:
\CMxlharpoon{X} \: \CMxlharpoon{x} \:
\CMxlharpoon{Y} \: \CMxlharpoon{y} \:
\CMxlharpoon{Z} \: \CMxlharpoon{z}
$ \\
$
\CMlrharpoon{X} \: \CMlrharpoon{x} \:
\CMlrharpoon{Y} \: \CMlrharpoon{y} \:
\CMlrharpoon{Z} \: \CMlrharpoon{z} \:
\CMxlrharpoon{X} \: \CMxlrharpoon{x} \:
\CMxlrharpoon{Y} \: \CMxlrharpoon{y} \:
\CMxlrharpoon{Z} \: \CMxlrharpoon{z}
$ \\





XyXyXyXyXyXyXyXyXyXyXyXyXyXyXyXyXyXyXyXy
XyXyXyXyXyXyXyXyXyXyXyXyXyXyXyXyXyXyXyXy
XyXyXyXyXyXyXyXyXyXyXyXyXyXyXyXyXyXyXyXy
XyXyXyXyXyXyXyXyXyXyXyXyXyXyXyXyXyXyXyXy
XyXyXyXyXyXyXyXyXyXyXyXyXyXyXyXyXyXyXyXy
XyXyXyXyXyXyXyXyXyXyXyXyXyXyXyXyXyXyXyXy
\myarrow{X}a$\xleftrightharpoonup{X} \CMxrarrow{Y} \vvvect{x}^+_{3:5} \vvvect{Y} \vvvect{A} \vvvect[shorten <= 3pt]{XY}$ XyXyXyXyXyXyXyXyXyXyXyXyXyXyXyXyXyXyXyXy
\\
XyXyXyXyXyXyXyXyXyXyXyXyXyXyXyXyXyXyXyXy
XyXyXyXyXyXyXyXyXyXyXyXyXyXyXyXyXyXyXyXy
XyXyXyXyXyXyXyXyXyXyXyXyXyXyXyXyXyXyXyXy
XyXyXyXyXyXyXyXyXyXyXyXyXyXyXyXyXyXyXyXy
XyXyXyXyXyXyXyXyXyXyXyXyXyXyXyXyXyXyXyXy
\\





$\overline{A}\texbox{$a$}\texbox{$b$}\texbox{$c$}\texbox{{}}\texbox{$\mkern5mu\overline{\texbox{$X$}}$}\texbox{abc}$\\
$\overline{A}\texbox{$a$}\texbox{$b$}\texbox{$c$}\texbox{{}}\texbox{$\mkern5mu\overline{\texbox{$\mkern-5muX$}}$}\texbox{abc}$\\
$\overline{A}abc{}\mkern5mu\overline{\mkern-5mu X}abc$\\
\texbox{$a$}\texbox{$b$}\texbox{$c$}\texbox{$\overline{\texbox{$X$}}$}\texbox{$a$}\texbox{$b$}\texbox{$c$} \\

$abc\overline{X}abc$\\
$      \Qwe(>),      % W with right-pointing harpoon
 \quad \Qaz(<),      % Q with left-pointing arrow
$ \\


abc\tikz[baseline,inner sep=0pt, outer sep=0pt]{\node[anchor=base] {d};}efg

DEMO: $\msf(>)[0][3]^3 \: \msd(>)[0][3]^3 \: \msh(>)[0][3]^3$\\

$\xleftrightharpoonup{X} \qquad \xleftrightharpoonup{XY} \qquad
\xleftrightharpoonup{x} \qquad \xleftrightharpoonup{i}$ \\

\blindtext $\xleftrightharpoonup{X} \vvect{X}\ : \vvvect{X} \: \vect{X}\vect{X}\vect{X}\vect{X}\vect{X}\vect{X} \: \vect{x} \vect{y}$ \blindtext

$\Qwe(<) \: \vect{W} \: \vect{WW} \: \vect{A} \: \vect{y} \: \vect{f} \: \vect{x} \quad \vect{X} \: \vect{i} \: \vect{I} \: \overrightarrow{X}$

\makeatletter

\begin{LTXexample}[pos=b]
\ttfamily\small\meaning\overrightarrow
\end{LTXexample}


\begin{LTXexample}[pos=b]
\makeatletter
\ttfamily\small\meaning\overarrow@
\makeatother
\end{LTXexample}

\begin{LTXexample}[pos=b]
\makeatletter
\ttfamily\small\meaning\rightarrowfill@
\makeatother
\end{LTXexample}

\begin{LTXexample}[pos=b]
\makeatletter
\ttfamily\small\meaning\arrowfill@
\makeatother
\end{LTXexample}


Make \verb+cmvec.sty+ available to your \LaTeX\
installation. A simple way to do this is to copy \verb+cmvec.sty+ into
the same directory as your source \LaTeX\ document. Then, add the following
to your preamble:
\begin{center}
\verb+\usepackage{cmvec}+
\end{center}

The \texttt{cmvec} package provides a number of macros, but mostly, there
are only two that you will need:

\begin{enumerate}
  \item \verb+\CMIndexedSymbol[arrowtype]{macroname}{symbol}+
  \item \verb+\CMSuperIndexedSymbol[arrowtype]{macroname}{symbol}{superscript}+
\end{enumerate}

Each of these macros defines a macro \verb+\macroname+ that supports
left, right, and leftright vector symbols and also provides Python-like
indexing.  These commands can be used anywhere in the document, but typically,
one should declare them just once in the preamble. Valid arrowtypes are:
\verb+arrow+ and \verb+harpoon+; the default value is \verb+harpoon+.
The macro that is defined by this command supports the following syntax:
\begin{center}
\verb+\macroname(direction)[start][end]+
\end{center}
All three options \verb+(direction)+, \verb+[start]+ \verb+[end]+ are
optional. Valid directions are \verb+<+, \verb+>+, \verb+<>+ for left, right,
and leftright directions. This is best explained through example:

\begin{LTXexample}[pos=b]
% Define the indexed symbol
\CMIndexedSymbol[harpoon]{MS}{X}

% Now use it
$\begin{matrix*}[l]
\MS & \MS[0] & \MS[0][L] & \MS[][L] & \MS[L][]\\
\MS(<) & \MS(>) & \MS(<>) & \MS(<)[0] & \MS(<>)[t]
\end{matrix*}$
\end{LTXexample}

Note that practically one will choose to use the indexing notation (as in
$X_{a:b}$) or the vector notation, but not both (with exception to adding a
time index for a semi-infinite sequence).\\

Suppose you plan on indexing both $X$ and $Y$, then you define both:
\begin{LTXexample}[pos=b]
\CMIndexedSymbol{MSi}{X} % input
\CMIndexedSymbol{MSo}{Y} % output
$\lim_{L \to \infty} I[ \MSi[0][L] : \MSo[0][L] ] \stackrel{?}{=} I[ \MSi(<)[t] : \MSo(>)[t]]$
\end{LTXexample}

It may also be helpful to freeze arrow directions to certain macro names:
\begin{LTXexample}[pos=b]
\CMIndexedSymbol{MS}{X}
\CMIndexedSymbol{ms}{x}
\newcommand{\Past}{\MS(<)}
\newcommand{\past}{\ms(<)}
\newcommand{\Future}{\MS(>)}
\newcommand{\future}{\ms(>)}
\begin{align*}
\Past[0] &= \cdots \MS[-3] \MS[-2] \MS[-1] &
\Future[0] &= \MS[0] \MS[1] \MS[2] \cdots \\
\past[0] &= \cdots \ms[-3] \ms[-2] \ms[-1] &
\future[0] &= \ms[0] \ms[1] \ms[2] \cdots \\
\end{align*}
\end{LTXexample}

That's pretty much it.\\

The package defines a number of other lower-level
commands that \emph{might} be of more general use, but probably not.
These are described now.

\begin{itemize}

\item Proper argmin and argmax:
\begin{LTXexample}[pos=b]
\begin{align*}
  &\arg\min_x (2x^2 - 3x + 5) \\
  &\argmin_x (2x^2 - 3x + 5)
\end{align*}
\end{LTXexample}

\item For summations with wide subscripts\ldots
\begin{LTXexample}[pos=b]
\begin{align*}
  A &= \sum_{i, j \in B_{ij}} X_i^j\\
  A &= \sum_{\mathclap{i, j \in B_{ij}}} X_i
\end{align*}
\end{LTXexample}

\item When you want to center math within a box whose width is
specified by other math.
\begin{LTXexample}[pos=b]
$aaabbbccc$\\
$aaa\phantomword[c]{bbb}{Q}ccc$
\end{LTXexample}

\item A customizable vector symbol. Macros should use this.
\begin{LTXexample}[pos=b]
\CMvector[symbol=\leftarrow]{X} \quad
\CMvector[symbol=\leftarrow, pre=\Large\textcolor{red}, ]{X} \quad
\CMvector[symbol=\leftarrow, post=*]{X} \quad
\CMvector[symbol=\leftarrow, raise=1.8]{X}
\end{LTXexample}

\item Although \verb+\leftharpoonup+ and \verb+\rightharpoonup+ exist, there
is no left-right harpoon. Here is a customized version that combines the
ones that do exist. The spacing is hardcoded and probably will only look good
with certain fonts.
\begin{LTXexample}[pos=b]
$\leftharpoonup \: \leftrightharpoonup \: \rightharpoonup$
\end{LTXexample}

\item Convenience functions that use \verb+\CMvector+.
\begin{LTXexample}[pos=b]
\CMlarrow{X} \: \CMlrarrow{X} \: \CMrarrow{X} \\
\CMlharpoon{X} \: \CMlrharpoon{X} \: \CMrharpoon{X} \\
\end{LTXexample}

\item A macro that defines specialized vector symbols that make use of Python index notation
and has clean support for the vector direction, depending on what notation you want to use.
\begin{LTXexample}[pos=b]
\CMIndexedSymbol{MS}{X}
\[\begin{matrix}
\MS     & \MS[3]     & \MS[3][5]     & \MS[][5]     & \MS[5][]     \\
\MS(<)  & \MS(<)[3]  & \MS(<)[3][5]  & \MS(<)[][5]  & \MS(<)[5][]  \\
\MS(>)  & \MS(>)[3]  & \MS(>)[3][5]  & \MS(>)[][5]  & \MS(>)[5][]  \\
\MS(<>) & \MS(<>)[3] & \MS(<>)[3][5] & \MS(<>)[][5] & \MS(<>)[5][] \\
\end{matrix}\]

\CMIndexedSymbol[arrow]{MS}{X}
\[\begin{matrix}
\MS     & \MS[3]     & \MS[3][5]     & \MS[][5]     & \MS[5][]     \\
\MS(<)  & \MS(<)[3]  & \MS(<)[3][5]  & \MS(<)[][5]  & \MS(<)[5][]  \\
\MS(>)  & \MS(>)[3]  & \MS(>)[3][5]  & \MS(>)[][5]  & \MS(>)[5][]  \\
\MS(<>) & \MS(<>)[3] & \MS(<>)[3][5] & \MS(<>)[][5] & \MS(<>)[5][] \\
\end{matrix}\]
\end{LTXexample}

\item Typically, you'll want to set up a few of these for regular use:
\begin{LTXexample}[pos=b]
% Put this in preamble somewhere
\CMIndexedSymbol[harpoon]{MS}{X}
\CMIndexedSymbol[harpoon]{ms}{x}
\CMSuperIndexedSymbol[arrow]{FCS}{S}{+}

% Some familiar commands
\newcommand{\BiInfinity}{\MS(<>)}
\newcommand{\biinfinity}{\ms(<>)}
\newcommand{\Past}{\MS(<)}
\newcommand{\past}{\ms(<)}
\newcommand{\Future}{\MS(>)}
\newcommand{\future}{\ms(>)}

% Now we can use them
\begin{displaymath}
\begin{matrix}
  \BiInfinity & \biinfinity & \Past & \past & \Future & \future \\
  \FCS[0] & \FCS(>)[3] & \MS[0][3] & \past[3] & \MS(<>)[3] & \Future[3]
\end{matrix}
\end{displaymath}
\end{LTXexample}

\item Notation for single symbol, range of symbols. Two different options
for semi-infinite sequences. If you might be using bi-infinite sequences,
it is recommended you use the second option for semi-infinite sequences.
\begin{LTXexample}[pos=b]
\CMIndexedSymbol[harpoon]{MS}{X}
\begin{align*}
  \MS         &&& \text{symbol}\\
  \MS[t]      &&& \text{symbol at time } t\\
  \MS[-1][3]  &&& \MS[-1] \MS[0] \MS[1] \MS[2]\\
  &&&\\
  \MS[][3]    &&& \cdots \MS[0] \MS[1] \MS[2] \\
  \MS[3][]    &&& \MS[3] \MS[4] \MS[5] \cdots\\
  &&&\\
  \MS(<)[3]   &&& \cdots \MS[0] \MS[1] \MS[2] \\
  \MS(>)[3]   &&& \MS[3] \MS[4] \MS[5] \cdots\\
  \MS(<>)[3]  &&& \MS(<)[3]\MS(>)[3]
\end{align*}
\end{LTXexample}

\end{itemize}

\end{document}
