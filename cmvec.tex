\documentclass{article}

\usepackage[margin=.5in]{geometry}

\usepackage{cmvec}

% For demoing code
\usepackage{showexpl}
\usepackage{xcolor}
\lstset{explpreset={
  rframe={},
  xleftmargin=.25in,
  columns=flexible,
  language={TeX},
  basicstyle=\ttfamily,
  backgroundcolor=\color{gray!30}
}}

\begin{document}
\setlength{\parindent}{0pt}

Make \verb+cmvec.sty+ available to your \LaTeX\
installation. A simple way to do this is to copy \verb+cmvec.sty+ into
the same directory as your source \LaTeX\ document. Then, add the following
to your preamble:
\begin{center}
\verb+\usepackage{cmvec}+
\end{center}

The \texttt{cmvec} package provides a number of macros, but mostly, there
are only two that you will need:

\begin{enumerate}
  \item \verb+\CMIndexedSymbol[arrowtype]{macroname}{symbol}+
  \item \verb+\CMSuperIndexedSymbol[arrowtype]{macroname}{symbol}{superscript}+
\end{enumerate}

Each of these macros defines a macro \verb+\macroname+ that supports
left, right, and leftright vector symbols and also provides Python-like
indexing.  These commands can be used anywhere in the document, but typically,
one should declare them just once in the preamble. Valid arrowtypes are:
\verb+arrow+ and \verb+harpoon+; the default value is \verb+harpoon+.
The macro that is defined by this command supports the following syntax:
\begin{center}
\verb+\macroname(direction)[start][end]+
\end{center}
All three options \verb+(direction)+, \verb+[start]+ \verb+[end]+ are
optional. Valid directions are \verb+<+, \verb+>+, \verb+<>+ for left, right,
and leftright directions. This is best explained through example:

\begin{LTXexample}[pos=b]
% Define the indexed symbol
\CMIndexedSymbol[harpoon]{MS}{X}

% Now use it
$\begin{matrix*}[l]
\MS & \MS[0] & \MS[0][L] & \MS[][L] & \MS[L][]\\
\MS(<) & \MS(>) & \MS(<>) & \MS(<)[0] & \MS(<>)[t]
\end{matrix*}$
\end{LTXexample}

Note that practically one will choose to use the indexing notation (as in
$X_{a:b}$) or the vector notation, but not both (with exception to adding a
time index for a semi-infinite sequence).\\

Suppose you plan on indexing both $X$ and $Y$, then you define both:
\begin{LTXexample}[pos=b]
\CMIndexedSymbol{MSi}{X} % input
\CMIndexedSymbol{MSo}{Y} % output
$\lim_{L \to \infty} I[ \MSi[0][L] : \MSo[0][L] ] \stackrel{?}{=} I[ \MSi(<)[t] : \MSo(>)[t]]$
\end{LTXexample}

It may also be helpful to freeze arrow directions to certain macro names:
\begin{LTXexample}[pos=b]
\CMIndexedSymbol{MS}{X}
\CMIndexedSymbol{ms}{x}
\newcommand{\Past}{\MS(<)}
\newcommand{\past}{\ms(<)}
\newcommand{\Future}{\MS(>)}
\newcommand{\future}{\ms(>)}
\begin{align*}
\Past[0] &= \cdots \MS[-3] \MS[-2] \MS[-1] &
\Future[0] &= \MS[0] \MS[1] \MS[2] \cdots \\
\past[0] &= \cdots \ms[-3] \ms[-2] \ms[-1] &
\future[0] &= \ms[0] \ms[1] \ms[2] \cdots \\
\end{align*}
\end{LTXexample}

That's pretty much it.\\

The package defines a number of other lower-level
commands that \emph{might} be of more general use, but probably not.
These are described now.

\begin{itemize}

\item Proper argmin and argmax:
\begin{LTXexample}[pos=b]
\begin{align*}
  &\arg\min_x (2x^2 - 3x + 5) \\
  &\argmin_x (2x^2 - 3x + 5)
\end{align*}
\end{LTXexample}

\item For summations with wide subscripts\ldots
\begin{LTXexample}[pos=b]
\begin{align*}
  A &= \sum_{i, j \in B_{ij}} X_i^j\\
  A &= \sum_{\mathclap{i, j \in B_{ij}}} X_i
\end{align*}
\end{LTXexample}

\item When you want to center math within a box whose width is
specified by other math.
\begin{LTXexample}[pos=b]
$aaabbbccc$\\
$aaa\phantomword[c]{bbb}{Q}ccc$
\end{LTXexample}

\item A customizable vector symbol. Macros should use this.
\begin{LTXexample}[pos=b]
\CMvector[symbol=\leftarrow]{X} \quad
\CMvector[symbol=\leftarrow, pre=\Large\textcolor{red}, ]{X} \quad
\CMvector[symbol=\leftarrow, post=*]{X} \quad
\CMvector[symbol=\leftarrow, raise=1.8]{X}
\end{LTXexample}

\item Although \verb+\leftharpoonup+ and \verb+\rightharpoonup+ exist, there
is no left-right harpoon. Here is a customized version that combines the
ones that do exist. The spacing is hardcoded and probably will only look good
with certain fonts.
\begin{LTXexample}[pos=b]
$\leftharpoonup \: \leftrightharpoonup \: \rightharpoonup$
\end{LTXexample}

\item Convenience functions that use \verb+\CMvector+.
\begin{LTXexample}[pos=b]
\CMlarrow{X} \: \CMlrarrow{X} \: \CMrarrow{X} \\
\CMlharpoon{X} \: \CMlrharpoon{X} \: \CMrharpoon{X} \\
\end{LTXexample}

\item A macro that defines specialized vector symbols that make use of Python index notation
and has clean support for the vector direction, depending on what notation you want to use.
\begin{LTXexample}[pos=b]
\CMIndexedSymbol{MS}{X}
\[\begin{matrix}
\MS     & \MS[3]     & \MS[3][5]     & \MS[][5]     & \MS[5][]     \\
\MS(<)  & \MS(<)[3]  & \MS(<)[3][5]  & \MS(<)[][5]  & \MS(<)[5][]  \\
\MS(>)  & \MS(>)[3]  & \MS(>)[3][5]  & \MS(>)[][5]  & \MS(>)[5][]  \\
\MS(<>) & \MS(<>)[3] & \MS(<>)[3][5] & \MS(<>)[][5] & \MS(<>)[5][] \\
\end{matrix}\]

\CMIndexedSymbol[arrow]{MS}{X}
\[\begin{matrix}
\MS     & \MS[3]     & \MS[3][5]     & \MS[][5]     & \MS[5][]     \\
\MS(<)  & \MS(<)[3]  & \MS(<)[3][5]  & \MS(<)[][5]  & \MS(<)[5][]  \\
\MS(>)  & \MS(>)[3]  & \MS(>)[3][5]  & \MS(>)[][5]  & \MS(>)[5][]  \\
\MS(<>) & \MS(<>)[3] & \MS(<>)[3][5] & \MS(<>)[][5] & \MS(<>)[5][] \\
\end{matrix}\]
\end{LTXexample}

\item Typically, you'll want to set up a few of these for regular use:
\begin{LTXexample}[pos=b]
% Put this in preamble somewhere
\CMIndexedSymbol[harpoon]{MS}{X}
\CMIndexedSymbol[harpoon]{ms}{x}
\CMSuperIndexedSymbol[arrow]{FCS}{S}{+}

% Some familiar commands
\newcommand{\BiInfinity}{\MS(<>)}
\newcommand{\biinfinity}{\ms(<>)}
\newcommand{\Past}{\MS(<)}
\newcommand{\past}{\ms(<)}
\newcommand{\Future}{\MS(>)}
\newcommand{\future}{\ms(>)}

% Now we can use them
\begin{displaymath}
\begin{matrix}
  \BiInfinity & \biinfinity & \Past & \past & \Future & \future \\
  \FCS[0] & \FCS(>)[3] & \MS[0][3] & \past[3] & \MS(<>)[3] & \Future[3]
\end{matrix}
\end{displaymath}
\end{LTXexample}

\item Notation for single symbol, range of symbols. Two different options
for semi-infinite sequences. If you might be using bi-infinite sequences,
it is recommended you use the second option for semi-infinite sequences.
\begin{LTXexample}[pos=b]
\CMIndexedSymbol[harpoon]{MS}{X}
\begin{align*}
  \MS         &&& \text{symbol}\\
  \MS[t]      &&& \text{symbol at time } t\\
  \MS[-1][3]   &&& \MS[-1] \MS[0] \MS[1] \MS[2]\\
  &&&\\
  \MS[][3]    &&& \cdots \MS[0] \MS[1] \MS[2] \\
  \MS[3][]    &&& \MS[3] \MS[4] \MS[5] \cdots\\
  &&&\\
  \MS(<)[3]   &&& \cdots \MS[0] \MS[1] \MS[2] \\
  \MS(>)[3]   &&& \MS[3] \MS[4] \MS[5] \cdots\\
  \MS(<>)[3]  &&& \MS(<)[3]\MS(>)[3]
\end{align*}
\end{LTXexample}

\end{itemize}

\end{document}
