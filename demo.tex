\documentclass{article}

\usepackage[margin=1in]{geometry}

\usepackage{amsfonts}
\usepackage{etoolbox}
\usepackage{mathtools}
\usepackage{showexpl}
\usepackage{xcolor}
\usepackage{xkeyval}

\lstset{explpreset={
  rframe={},
  xleftmargin=.25in,
  columns=flexible,
  language={TeX},
  basicstyle=\ttfamily,
  backgroundcolor=\color{gray!30}
}}

% Requires:
%   \usepackage{etoolbox}
%   \usepackage{mathtools}
%   \usepackage{xkeyval}

\DeclareMathOperator*{\argmin}{argmin}
\DeclareMathOperator*{\argmax}{argmax}

% Useful for seeing how TeX does its spacing and alignment.
\newcommand{\texbox}[1]{{\setlength{\fboxsep}{0pt}\fbox{#1}}}


% TUGboat, Volume 0 (2001), No. 0
% http://math.arizona.edu/~aprl/publications/mathclap/perlis_mathclap_24Jun2003.pdf
% For comparison, here are the existing overlap macros:
% \def\llap#1{\hbox to 0pt{\hss#1}}
% \def\rlap#1{\hbox to 0pt{#1\hss}}
%
% Not needed if you use \usepackage{mathtools}.
%
\def\clap#1{\hbox to 0pt{\hss#1\hss}}
\def\mathllap{\mathpalette\mathllapinternal}
\def\mathrlap{\mathpalette\mathrlapinternal}
\def\mathclap{\mathpalette\mathclapinternal}
\def\mathllapinternal#1#2{%
\llap{$\mathsurround=0pt#1{#2}$}}
\def\mathrlapinternal#1#2{%
\rlap{$\mathsurround=0pt#1{#2}$}}
\def\mathclapinternal#1#2{%
\clap{$\mathsurround=0pt#1{#2}$}}


% Usage:  \phantomword[c]{hiddenmath}{shownmath}
%
% The hidden text defines the box size.
% The shown text is placed inside the box.
% The optional argument is the alignment: l,c,r
\MHInternalSyntaxOn
% Using mathpalette requires more shuffling of arguments
\providecommand*\phantomword[3][c]{%
  \mathchoice
  {\MT_phantom_word:NNnn #1\displaystyle {#2}{#3}}%
  {\MT_phantom_word:NNnn #1\textstyle {#2}{#3}}%
  {\MT_phantom_word:NNnn #1\scriptstyle {#2}{#3}}%
  {\MT_phantom_word:NNnn #1\scriptscriptstyle {#2}{#3}}%
}
\def\MT_phantom_word:NNnn #1#2#3#4{%
  \@begin@tempboxa\hbox{$\m@th#2#4$}%
% can't use \settowidth as that also uses \@tempboxa...
    \setlength\@tempdima{\widthof{$\m@th#2#3$}}%
    \hbox{\hb@xt@\@tempdima{\csname bm@#1\endcsname}}%
  \@end@tempboxa}
\MHInternalSyntaxOff





\makeatletter
\newlength\CMlength
\define@cmdkey[CM]{vector}[CM@@]{raise}{}
\define@cmdkey[CM]{vector}[CM@@]{pre}{}
\define@cmdkey[CM]{vector}[CM@@]{post}{}
\define@cmdkey[CM]{vector}[CM@@]{symbol}{}
\presetkeys[CM]{vector}{%
  raise=1.02,%
  pre={\scriptscriptstyle\,},%
  post={}%
}{}
\newcommand{\CMvector}[2][]{%
  \setkeys[CM]{vector}[]{#1}%
  \settoheight{\CMlength}{\ensuremath{#2}}%
  \setlength{\CMlength}{\CM@@raise\CMlength}%
  \ensuremath{%
    \mathrlap{\ensuremath{#2}}%
    \smash{\phantomword[c]%
      {\ensuremath{#2}}%
      {\raise \CMlength \hbox{\ensuremath{\CM@@pre\CM@@symbol\CM@@post}}}%
    }%
  }%
}
\makeatother

%%% Seems to give better spacing
%%% If you change it, test against:
%%%    $\CMlrharpoon{A}\CMlrharpoon{W}\CMlrharpoon{X}\CMlrharpoon{Y}\CMlrharpoon{Z}\CMlrharpoon{I}\CMlrharpoon{O}$
\newcommand{\leftrightharpoonup}{\mathrlap{\leftharpoonup}\phantomword[l]{\,\leftharpoonup}{\,\rightharpoonup}}

%% arrows
\newcommand{\CMlarrow}[1]{\CMvector[symbol=\leftarrow]{#1}}
\newcommand{\CMrarrow}[1]{\CMvector[symbol=\rightarrow]{#1}}
\newcommand{\CMlrarrow}[1]{\CMvector[symbol=\leftrightarrow]{#1}}
%% harpoons
\newcommand{\CMlharpoon}[1]{\CMvector[symbol=\leftharpoonup,pre=\scriptscriptstyle]{#1}}
\newcommand{\CMrharpoon}[1]{\CMvector[symbol=\rightharpoonup,pre=\scriptscriptstyle]{#1}}
\newcommand{\CMlrharpoon}[1]{\CMvector[symbol=\leftrightharpoonup,pre=\scriptscriptstyle]{#1}}




\makeatletter
\def\BEsep{:}
% \CMIndexedSymbol[type]{cmdname}{symbol}
\newcommand{\CMIndexedSymbol}[3][harpoon]{%
  \ifstrequal{#1}{arrow}
    {% Arrows
      \expandafter\let\csname Lobject#2\endcsname\CMlarrow
      \expandafter\let\csname Robject#2\endcsname\CMrarrow
      \expandafter\let\csname LRobject#2\endcsname\CMlrarrow
    }
    {
      \expandafter\let\csname Lobject#2\endcsname\CMlharpoon
      \expandafter\let\csname Robject#2\endcsname\CMrharpoon
      \expandafter\let\csname LRobject#2\endcsname\CMlrharpoon
    }
  \protected\expandafter\def\csname #2\endcsname{%
    \@ifnextchar(%)
      {\csname #2@i\endcsname}
      {\csname #2@i\endcsname()}%
  }%
  \expandafter\def\csname #2@i\endcsname(##1){%
    % figure out what the main character should be
    \ifstrequal{##1}{>}%
      {\def\CMtemp{\csname Robject#2\endcsname{#3}}}% (>) --> right
      {% false
        \ifstrequal{##1}{<}%
          {\def\CMtemp{\csname Lobject#2\endcsname{#3}}}% (<) --> left
          {%
            \ifstrequal{##1}{<>}%
              {\def\CMtemp{\csname LRobject#2\endcsname{#3}}}% (<>) --> left,right
              {\def\CMtemp{#3}}% () or anything else --> regular
          }%
      }%
    \@ifnextchar[%]
      {\csname #2@ii\endcsname(\CMtemp)}%
      {\CMtemp}%                                   \cmdname
  }%
  \expandafter\def\csname #2@ii\endcsname(##1)[##2]{%
    \@ifnextchar[%]
      {\csname #2@iii\endcsname({##1})[{##2}]}%
      {##1_{##2}}%                                  \cmdname[i]
  }%
  \expandafter\def\csname #2@iii\endcsname(##1)[##2][##3]{%
    ##1_{##2\BEsep##3}%                                        \cmdname[i][j]
  }%
}

% \CMSuperIndexedSymbol{cmdname}{symbol}{superscript}
\newcommand{\CMSuperIndexedSymbol}[4][harpoon]{%
  \ifstrequal{#1}{arrow}
    {% Arrows
      \expandafter\let\csname Lobject#2\endcsname\CMlarrow
      \expandafter\let\csname Robject#2\endcsname\CMrarrow
      \expandafter\let\csname LRobject#2\endcsname\CMlrarrow
    }
    {
      \expandafter\let\csname Lobject#2\endcsname\CMlharpoon
      \expandafter\let\csname Robject#2\endcsname\CMrharpoon
      \expandafter\let\csname LRobject#2\endcsname\CMlrharpoon
    }
  \protected\expandafter\def\csname #2\endcsname{%
    \@ifnextchar(%)
      {\csname #2@i\endcsname}%
      {\csname #2@i\endcsname()}%
  }%
  \expandafter\def\csname #2@i\endcsname(##1){%
    % figure out what the main character should be
    \ifstrequal{##1}{>}%
      {\def\CMtemp{\csname Robject#2\endcsname{#3}}}% (>) --> right
      {% false
        \ifstrequal{##1}{<}%
          {\def\CMtemp{\csname Lobject#2\endcsname{#3}}}% (<) --> left
          {%
            \ifstrequal{##1}{<>}%
              {\def\CMtemp{\csname LRobject#2\endcsname{#3}}}% (<>) --> left,right
              {\def\CMtemp{#3}}% () or anything else --> regular
          }%
      }%
    \@ifnextchar[%]
      {\csname #2@ii\endcsname(\CMtemp)}%
      {\CMtemp^{#4}}%                                   \cmdname
  }%
  \expandafter\def\csname #2@ii\endcsname(##1)[##2]{%
    \@ifnextchar[%]
      {\csname #2@iii\endcsname({##1})[{##2}]}%
      {##1_{##2}^{#4}}%                                  \cmdname[i]
  }%
  \expandafter\def\csname #2@iii\endcsname(##1)[##2][##3]{%
    ##1_{##2\BEsep##3}^{#4}%                                        \cmdname[i][j]
  }%
}

\makeatother


\begin{document}
\setlength{\parindent}{0pt}
\begin{itemize}


\item Proper argmin and argmax:
\begin{LTXexample}[pos=b]
\begin{align*}
  &\arg\min_x (2x^2 - 3x + 5) \\
  &\argmin_x (2x^2 - 3x + 5)
\end{align*}
\end{LTXexample}

\item For summations with wide subscripts\ldots
\begin{LTXexample}[pos=b]
\begin{align*}
  A &= \sum_{i, j \in B_{ij}} X_i^j\\
  A &= \sum_{\mathclap{i, j \in B_{ij}}} X_i
\end{align*}
\end{LTXexample}

\item When you want to center math within a box whose width is
specified by other math.
\begin{LTXexample}[pos=b]
$aaabbbccc$\\
$aaa\phantomword[c]{bbb}{Q}ccc$
\end{LTXexample}

\item A customizable vector symbol. Macros should use this.
\begin{LTXexample}[pos=b]
\CMvector[symbol=\leftarrow]{X} \quad
\CMvector[symbol=\leftarrow, pre=\Large\textcolor{red}, ]{X} \quad
\CMvector[symbol=\leftarrow, post=*]{X} \quad
\CMvector[symbol=\leftarrow, raise=1.8]{X}
\end{LTXexample}

\item Although \verb+\leftharpoonup+ and \verb+\rightharpoonup+ exist, there
is no left-right harpoon. Here is a customized version that combines the
ones that do exist. The spacing is hardcoded and probably will only look good
with certain fonts.
\begin{LTXexample}[pos=b]
$\leftharpoonup \: \leftrightharpoonup \: \rightharpoonup$
\end{LTXexample}

\item Convenience functions that use \verb+\CMvector+.
\begin{LTXexample}[pos=b]
\CMlarrow{X} \: \CMlrarrow{X} \: \CMrarrow{X} \\
\CMlharpoon{X} \: \CMlrharpoon{X} \: \CMrharpoon{X} \\
\end{LTXexample}

\item A macro that defines specialized vector symbols that make use of Python index notation
and has clean support for the vector direction, depending on what notation you want to use.
\begin{LTXexample}[pos=b]
\CMIndexedSymbol{MS}{X}
\[\begin{matrix}
\MS     & \MS[3]     & \MS[3][5]     & \MS[][5]     & \MS[5][]     \\
\MS(<)  & \MS(<)[3]  & \MS(<)[3][5]  & \MS(<)[][5]  & \MS(<)[5][]  \\
\MS(>)  & \MS(>)[3]  & \MS(>)[3][5]  & \MS(>)[][5]  & \MS(>)[5][]  \\
\MS(<>) & \MS(<>)[3] & \MS(<>)[3][5] & \MS(<>)[][5] & \MS(<>)[5][] \\
\end{matrix}\]

\CMIndexedSymbol[arrow]{MS}{X}
\[\begin{matrix}
\MS     & \MS[3]     & \MS[3][5]     & \MS[][5]     & \MS[5][]     \\
\MS(<)  & \MS(<)[3]  & \MS(<)[3][5]  & \MS(<)[][5]  & \MS(<)[5][]  \\
\MS(>)  & \MS(>)[3]  & \MS(>)[3][5]  & \MS(>)[][5]  & \MS(>)[5][]  \\
\MS(<>) & \MS(<>)[3] & \MS(<>)[3][5] & \MS(<>)[][5] & \MS(<>)[5][] \\
\end{matrix}\]
\end{LTXexample}

\item Typically, you'll want to set up a few of these for regular use:
\begin{LTXexample}[pos=b]
% Put this in preamble somewhere
\CMIndexedSymbol[harpoon]{MS}{X}
\CMIndexedSymbol[harpoon]{ms}{x}
\CMSuperIndexedSymbol[arrow]{FCS}{S}{+}

% Some familiar commands
\newcommand{\BiInfinity}{\MS(<>)}
\newcommand{\biinfinity}{\ms(<>)}
\newcommand{\Past}{\MS(<)}
\newcommand{\past}{\ms(<)}
\newcommand{\Future}{\MS(>)}
\newcommand{\future}{\ms(>)}

% Now we can use them
\begin{displaymath}
\begin{matrix}
  \BiInfinity & \biinfinity & \Past & \past & \Future & \future \\
  \FCS[0] & \FCS(>)[3] & \MS[0][3] & \past[3] & \MS(<>)[3] & \Future[3]
\end{matrix}
\end{displaymath}
\end{LTXexample}

\item Notation for single symbol, range of symbols. Two different options
for semi-infinite sequences. If you might be using bi-infinite sequences,
it is recommended you use the second option for semi-infinite sequences.
\begin{LTXexample}[pos=b]
\CMIndexedSymbol[harpoon]{MS}{X}
\begin{align*}
  \MS         &&& \text{symbol}\\
  \MS[t]      &&& \text{symbol at time } t\\
  \MS[-1][3]   &&& \MS[-1] \MS[0] \MS[1] \MS[2]\\
  &&&\\
  \MS[][3]    &&& \cdots \MS[0] \MS[1] \MS[2] \\
  \MS[3][]    &&& \MS[3] \MS[4] \MS[5] \cdots\\
  &&&\\
  \MS(<)[3]   &&& \cdots \MS[0] \MS[1] \MS[2] \\
  \MS(>)[3]   &&& \MS[3] \MS[4] \MS[5] \cdots\\
  \MS(<>)[3]  &&& \MS(<)[3]\MS(>)[3]
\end{align*}
\end{LTXexample}





\end{itemize}

\end{document}
