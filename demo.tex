\documentclass{article}

\usepackage[margin=1in]{geometry}

\usepackage{amsfonts}
\usepackage{etoolbox}
\usepackage{mathtools}
\usepackage{showexpl}
\usepackage{xcolor}
\usepackage{xkeyval}

\lstset{explpreset={
  rframe={},
  xleftmargin=.25in,
  columns=flexible,
  language={TeX},
  basicstyle=\ttfamily,
  backgroundcolor=\color{gray!30}
}}

\input{macros}

\begin{document}
\setlength{\parindent}{0pt}
\begin{itemize}


\item Proper argmin and argmax:
\begin{LTXexample}[pos=b]
\begin{align*}
  &\arg\min_x (2x^2 - 3x + 5) \\
  &\argmin_x (2x^2 - 3x + 5)
\end{align*}
\end{LTXexample}

\item For summations with wide subscripts\ldots
\begin{LTXexample}[pos=b]
\begin{align*}
  A &= \sum_{i, j \in B_{ij}} X_i^j\\
  A &= \sum_{\mathclap{i, j \in B_{ij}}} X_i
\end{align*}
\end{LTXexample}

\item When you want to center math within a box whose width is
specified by other math.
\begin{LTXexample}[pos=b]
$aaabbbccc$\\
$aaa\phantomword[c]{bbb}{Q}ccc$
\end{LTXexample}

\item A customizable vector symbol. Macros should use this.
\begin{LTXexample}[pos=b]
\CMvector[symbol=\leftarrow]{X} \quad
\CMvector[symbol=\leftarrow, pre=\Large\textcolor{red}, ]{X} \quad
\CMvector[symbol=\leftarrow, post=*]{X} \quad
\CMvector[symbol=\leftarrow, raise=1.8]{X}
\end{LTXexample}

\item Although \verb+\leftharpoonup+ and \verb+\rightharpoonup+ exist, there
is no left-right harpoon. Here is a customized version that combines the
ones that do exist. The spacing is hardcoded and probably will only look good
with certain fonts.
\begin{LTXexample}[pos=b]
$\leftharpoonup \: \leftrightharpoonup \: \rightharpoonup$
\end{LTXexample}

\item Convenience functions that use \verb+\CMvector+.
\begin{LTXexample}[pos=b]
\CMlarrow{X} \: \CMlrarrow{X} \: \CMrarrow{X} \\
\CMlharpoon{X} \: \CMlrharpoon{X} \: \CMrharpoon{X} \\
\end{LTXexample}

\item A macro that defines specialized vector symbols that make use of Python index notation
and has clean support for the vector direction, depending on what notation you want to use.
\begin{LTXexample}[pos=b]
\CMIndexedSymbol{MS}{X}
\[\begin{matrix}
\MS     & \MS[3]     & \MS[3][5]     & \MS[][5]     & \MS[5][]     \\
\MS(<)  & \MS(<)[3]  & \MS(<)[3][5]  & \MS(<)[][5]  & \MS(<)[5][]  \\
\MS(>)  & \MS(>)[3]  & \MS(>)[3][5]  & \MS(>)[][5]  & \MS(>)[5][]  \\
\MS(<>) & \MS(<>)[3] & \MS(<>)[3][5] & \MS(<>)[][5] & \MS(<>)[5][] \\
\end{matrix}\]

\CMIndexedSymbol[arrow]{MS}{X}
\[\begin{matrix}
\MS     & \MS[3]     & \MS[3][5]     & \MS[][5]     & \MS[5][]     \\
\MS(<)  & \MS(<)[3]  & \MS(<)[3][5]  & \MS(<)[][5]  & \MS(<)[5][]  \\
\MS(>)  & \MS(>)[3]  & \MS(>)[3][5]  & \MS(>)[][5]  & \MS(>)[5][]  \\
\MS(<>) & \MS(<>)[3] & \MS(<>)[3][5] & \MS(<>)[][5] & \MS(<>)[5][] \\
\end{matrix}\]
\end{LTXexample}

\item Typically, you'll want to set up a few of these for regular use:
\begin{LTXexample}[pos=b]
% Put this in preamble somewhere
\CMIndexedSymbol[harpoon]{MS}{X}
\CMIndexedSymbol[harpoon]{ms}{x}
\CMSuperIndexedSymbol[arrow]{FCS}{S}{+}

% Some familiar commands
\newcommand{\BiInfinity}{\MS(<>)}
\newcommand{\biinfinity}{\ms(<>)}
\newcommand{\Past}{\MS(<)}
\newcommand{\past}{\ms(<)}
\newcommand{\Future}{\MS(>)}
\newcommand{\future}{\ms(>)}

% Now we can use them
\begin{displaymath}
\begin{matrix}
  \BiInfinity & \biinfinity & \Past & \past & \Future & \future \\
  \FCS[0] & \FCS(>)[3] & \MS[0][3] & \past[3] & \MS(<>)[3] & \Future[3]
\end{matrix}
\end{displaymath}
\end{LTXexample}

\item Notation for single symbol, range of symbols. Two different options
for semi-infinite sequences. If you might be using bi-infinite sequences,
it is recommended you use the second option for semi-infinite sequences.
\begin{LTXexample}[pos=b]
\CMIndexedSymbol[harpoon]{MS}{X}
\begin{align*}
  \MS         &&& \text{symbol}\\
  \MS[t]      &&& \text{symbol at time } t\\
  \MS[-1][3]   &&& \MS[-1] \MS[0] \MS[1] \MS[2]\\
  &&&\\
  \MS[][3]    &&& \cdots \MS[0] \MS[1] \MS[2] \\
  \MS[3][]    &&& \MS[3] \MS[4] \MS[5] \cdots\\
  &&&\\
  \MS(<)[3]   &&& \cdots \MS[0] \MS[1] \MS[2] \\
  \MS(>)[3]   &&& \MS[3] \MS[4] \MS[5] \cdots\\
  \MS(<>)[3]  &&& \MS(<)[3]\MS(>)[3]
\end{align*}
\end{LTXexample}





\end{itemize}

\end{document}
