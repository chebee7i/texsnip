% Requires:
%   \usepackage{etoolbox}
%   \usepackage{mathtools}
%   \usepackage{xkeyval}

\DeclareMathOperator*{\argmin}{argmin}
\DeclareMathOperator*{\argmax}{argmax}

% Useful for seeing how TeX does its spacing and alignment.
\newcommand{\texbox}[1]{{\setlength{\fboxsep}{0pt}\fbox{#1}}}


% TUGboat, Volume 0 (2001), No. 0
% http://math.arizona.edu/~aprl/publications/mathclap/perlis_mathclap_24Jun2003.pdf
% For comparison, here are the existing overlap macros:
% \def\llap#1{\hbox to 0pt{\hss#1}}
% \def\rlap#1{\hbox to 0pt{#1\hss}}
%
% Not needed if you use \usepackage{mathtools}.
%
\def\clap#1{\hbox to 0pt{\hss#1\hss}}
\def\mathllap{\mathpalette\mathllapinternal}
\def\mathrlap{\mathpalette\mathrlapinternal}
\def\mathclap{\mathpalette\mathclapinternal}
\def\mathllapinternal#1#2{%
\llap{$\mathsurround=0pt#1{#2}$}}
\def\mathrlapinternal#1#2{%
\rlap{$\mathsurround=0pt#1{#2}$}}
\def\mathclapinternal#1#2{%
\clap{$\mathsurround=0pt#1{#2}$}}


% Usage:  \phantomword[c]{hiddenmath}{shownmath}
%
% The hidden text defines the box size.
% The shown text is placed inside the box.
% The optional argument is the alignment: l,c,r
\MHInternalSyntaxOn
% Using mathpalette requires more shuffling of arguments
\providecommand*\phantomword[3][c]{%
  \mathchoice
  {\MT_phantom_word:NNnn #1\displaystyle {#2}{#3}}%
  {\MT_phantom_word:NNnn #1\textstyle {#2}{#3}}%
  {\MT_phantom_word:NNnn #1\scriptstyle {#2}{#3}}%
  {\MT_phantom_word:NNnn #1\scriptscriptstyle {#2}{#3}}%
}
\def\MT_phantom_word:NNnn #1#2#3#4{%
  \@begin@tempboxa\hbox{$\m@th#2#4$}%
% can't use \settowidth as that also uses \@tempboxa...
    \setlength\@tempdima{\widthof{$\m@th#2#3$}}%
    \hbox{\hb@xt@\@tempdima{\csname bm@#1\endcsname}}%
  \@end@tempboxa}
\MHInternalSyntaxOff





\makeatletter
\newlength\CMlength
\define@cmdkey[CM]{vector}[CM@@]{raise}{}
\define@cmdkey[CM]{vector}[CM@@]{pre}{}
\define@cmdkey[CM]{vector}[CM@@]{post}{}
\define@cmdkey[CM]{vector}[CM@@]{symbol}{}
\presetkeys[CM]{vector}{%
  raise=1.02,%
  pre={\scriptscriptstyle\,},%
  post={}%
}{}
\newcommand{\CMvector}[2][]{%
  \setkeys[CM]{vector}[]{#1}%
  \settoheight{\CMlength}{\ensuremath{#2}}%
  \setlength{\CMlength}{\CM@@raise\CMlength}%
  \ensuremath{%
    \mathrlap{\ensuremath{#2}}%
    \smash{\phantomword[c]%
      {\ensuremath{#2}}%
      {\raise \CMlength \hbox{\ensuremath{\CM@@pre\CM@@symbol\CM@@post}}}%
    }%
  }%
}
\makeatother

%%% Seems to give better spacing
%%% If you change it, test against:
%%%    $\CMlrharpoon{A}\CMlrharpoon{W}\CMlrharpoon{X}\CMlrharpoon{Y}\CMlrharpoon{Z}\CMlrharpoon{I}\CMlrharpoon{O}$
\newcommand{\leftrightharpoonup}{\mathrlap{\leftharpoonup}\phantomword[l]{\,\leftharpoonup}{\,\rightharpoonup}}

%% arrows
\newcommand{\CMlarrow}[1]{\CMvector[symbol=\leftarrow]{#1}}
\newcommand{\CMrarrow}[1]{\CMvector[symbol=\rightarrow]{#1}}
\newcommand{\CMlrarrow}[1]{\CMvector[symbol=\leftrightarrow]{#1}}
%% harpoons
\newcommand{\CMlharpoon}[1]{\CMvector[symbol=\leftharpoonup,pre=\scriptscriptstyle]{#1}}
\newcommand{\CMrharpoon}[1]{\CMvector[symbol=\rightharpoonup,pre=\scriptscriptstyle]{#1}}
\newcommand{\CMlrharpoon}[1]{\CMvector[symbol=\leftrightharpoonup,pre=\scriptscriptstyle]{#1}}




\makeatletter
\def\BEsep{:}
% \CMIndexedSymbol[type]{cmdname}{symbol}
\newcommand{\CMIndexedSymbol}[3][harpoon]{%
  \ifstrequal{#1}{arrow}
    {% Arrows
      \expandafter\let\csname Lobject#2\endcsname\CMlarrow
      \expandafter\let\csname Robject#2\endcsname\CMrarrow
      \expandafter\let\csname LRobject#2\endcsname\CMlrarrow
    }
    {
      \expandafter\let\csname Lobject#2\endcsname\CMlharpoon
      \expandafter\let\csname Robject#2\endcsname\CMrharpoon
      \expandafter\let\csname LRobject#2\endcsname\CMlrharpoon
    }
  \protected\expandafter\def\csname #2\endcsname{%
    \@ifnextchar(%)
      {\csname #2@i\endcsname}
      {\csname #2@i\endcsname()}%
  }%
  \expandafter\def\csname #2@i\endcsname(##1){%
    % figure out what the main character should be
    \ifstrequal{##1}{>}%
      {\def\CMtemp{\csname Robject#2\endcsname{#3}}}% (>) --> right
      {% false
        \ifstrequal{##1}{<}%
          {\def\CMtemp{\csname Lobject#2\endcsname{#3}}}% (<) --> left
          {%
            \ifstrequal{##1}{<>}%
              {\def\CMtemp{\csname LRobject#2\endcsname{#3}}}% (<>) --> left,right
              {\def\CMtemp{#3}}% () or anything else --> regular
          }%
      }%
    \@ifnextchar[%]
      {\csname #2@ii\endcsname(\CMtemp)}%
      {\CMtemp}%                                   \cmdname
  }%
  \expandafter\def\csname #2@ii\endcsname(##1)[##2]{%
    \@ifnextchar[%]
      {\csname #2@iii\endcsname({##1})[{##2}]}%
      {##1_{##2}}%                                  \cmdname[i]
  }%
  \expandafter\def\csname #2@iii\endcsname(##1)[##2][##3]{%
    ##1_{##2\BEsep##3}%                                        \cmdname[i][j]
  }%
}

% \CMSuperIndexedSymbol{cmdname}{symbol}{superscript}
\newcommand{\CMSuperIndexedSymbol}[4][harpoon]{%
  \ifstrequal{#1}{arrow}
    {% Arrows
      \expandafter\let\csname Lobject#2\endcsname\CMlarrow
      \expandafter\let\csname Robject#2\endcsname\CMrarrow
      \expandafter\let\csname LRobject#2\endcsname\CMlrarrow
    }
    {
      \expandafter\let\csname Lobject#2\endcsname\CMlharpoon
      \expandafter\let\csname Robject#2\endcsname\CMrharpoon
      \expandafter\let\csname LRobject#2\endcsname\CMlrharpoon
    }
  \protected\expandafter\def\csname #2\endcsname{%
    \@ifnextchar(%)
      {\csname #2@i\endcsname}%
      {\csname #2@i\endcsname()}%
  }%
  \expandafter\def\csname #2@i\endcsname(##1){%
    % figure out what the main character should be
    \ifstrequal{##1}{>}%
      {\def\CMtemp{\csname Robject#2\endcsname{#3}}}% (>) --> right
      {% false
        \ifstrequal{##1}{<}%
          {\def\CMtemp{\csname Lobject#2\endcsname{#3}}}% (<) --> left
          {%
            \ifstrequal{##1}{<>}%
              {\def\CMtemp{\csname LRobject#2\endcsname{#3}}}% (<>) --> left,right
              {\def\CMtemp{#3}}% () or anything else --> regular
          }%
      }%
    \@ifnextchar[%]
      {\csname #2@ii\endcsname(\CMtemp)}%
      {\CMtemp^{#4}}%                                   \cmdname
  }%
  \expandafter\def\csname #2@ii\endcsname(##1)[##2]{%
    \@ifnextchar[%]
      {\csname #2@iii\endcsname({##1})[{##2}]}%
      {##1_{##2}^{#4}}%                                  \cmdname[i]
  }%
  \expandafter\def\csname #2@iii\endcsname(##1)[##2][##3]{%
    ##1_{##2\BEsep##3}^{#4}%                                        \cmdname[i][j]
  }%
}

\makeatother
